% !TEX root = ../thesis.tex
%
\chapter{Introduction}
\label{sec:intro}

\chapterprecishere{"Some people worry that artificial intelligence will make us feel inferior, but then, anybody in his right mind should have an inferiority complex every time he looks at a flower."\par\raggedleft--- \textup{Alan Kay}, (Computer Scientist)}


https://sokogskriv.no/en/writing/structure/structuring-a-thesis/
http://www.charleslipson.com/How-to-write-a-thesis.htm

\section{Motivation and Problem Statement}
\label{sec:intro:motivation}
In the last decade may different solutions for \gls{nn} have been implemented, whereas these implementations propose various changes like the amount and distribution of connections between neurons, the weight calculations between neuronal connections or the amount of neuronal layers \todo{xor problem} of the network as well as other structural decisions. 
The efficiency of these algorithms depend on the problem space, which they were tested on. For example ....\todo{concrete examples}
\begin{enumerate}
	\item einleitung komplexe aufgaben
	\item bisher: nn networkd
	\item genetic algorithms
	\item nn + ga => neat and others
	\item \url{https://www.reddit.com/r/NeuralNetwork/comments/4nea5i/how_to_decide_what_neural_network_architecture_to}
	\item One complex problem are diverse games. 
	\item different games and criteria
	\item different algorithms and neat
	\item expectations of this bachelor work
\end{enumerate}


\section{Results}
\label{sec:intro:results}
\begin{enumerate}
	\item what was interesting to see
	\item contrast to expectations
\end{enumerate}


\subsection{Some References}
\label{sec:intro:results:refs}

\section{Thesis Structure}
\label{sec:intro:structure}

\textbf{Chapter \ref{sec:related}} \\[0.2em]

\textbf{Chapter \ref{sec:analysis}} \\[0.2em]

\textbf{Chapter \ref{sec:compare}} \\[0.2em]

\textbf{Chapter \ref{sec:conclusion}} \\[0.2em]
